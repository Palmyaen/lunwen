\chapter*{Abstract}

Industrial fault diagnosis represents a critical challenge in modern manufacturing systems, where early and accurate detection of equipment failures can prevent costly downtime and safety incidents. This thesis presents a novel hybrid Transformer-LSTM architecture for industrial fault diagnosis that combines the global attention mechanisms of Transformers with the sequential modeling capabilities of LSTM networks.

The proposed approach addresses the limitations of traditional fault diagnosis methods by effectively capturing both short-term fluctuations and long-term temporal dependencies in multi-dimensional industrial sensor data. The hybrid architecture employs a sophisticated feature fusion mechanism that leverages the complementary strengths of both neural network components while maintaining computational efficiency for real-time applications.

Extensive experimental evaluation on industrial datasets demonstrates the effectiveness of the proposed method, achieving 85.2\% classification accuracy across 11 different fault categories including backlash errors, bearing failures, and jerk motions. This performance significantly surpasses traditional approaches, with improvements of 6.9\% over pure LSTM networks, 3.5\% over pure Transformer architectures, and substantial gains over classical machine learning methods.

The system meets critical industrial requirements with an average inference time of 67ms per sample, enabling real-time fault detection and diagnosis. The modular design facilitates easy deployment and adaptation to different industrial environments and fault types.

This research contributes to the advancement of intelligent manufacturing systems and Industry 4.0 applications, providing a robust foundation for next-generation industrial fault diagnosis solutions. The hybrid approach opens new possibilities for combining different deep learning architectures to address complex temporal pattern recognition challenges in industrial automation.
